\chapter{Επισκόπηση της Ερευνητικής Περιοχής}

Η αυτονομία, η ανθεκτικότητα, αλλά και η παρατηρησιμότητα σε μεγάλης κλίμακας real time συστήματα αποτελούν βασικές απαιτήσεις για την υποστήριξη κρίσιμων εφαρμογών, ειδικά στον τομέα του Internet of Things (IoT). Καθώς η ποσότητα των παραγόμενων δεδομένων αυξάνεται εκθετικά, οι υποδομές που επεξεργάζονται και αποθηκεύουν αυτά τα δεδομένα οφείλουν να είναι όχι μόνο αποδοτικές, αλλά και αυτοπροσαρμοζόμενες, με δυνατότητα επιτήρησης και γρήγορης αποκατάστασης σε περιπτώσεις σφαλμάτων ή αστοχιών. Οι σύγχρονες cloud-native τεχνολογίες παρέχουν τα εργαλεία και τα πρότυπα για την υλοποίηση τέτοιων χαρακτηριστικών.

Παρότι υπάρχουν επιμέρους τεχνολογίες που προσφέρουν caching, observability ή self-healing δυνατότητες, η ενσωμάτωσή τους σε πραγματικές πολυεπίπεδες IoT πλατφόρμες συνοδεύεται από σημαντικές προκλήσεις. Στα κατανεμημένα συστήματα, η προσθήκη μιας νέας υπηρεσίας ή λειτουργικότητας συνοδεύεται συχνά από μη προβλέψιμες επιπτώσεις στην απόδοση του συστήματος. Αν και οι μηχανισμοί όπως το caching και το observability αποσκοπούν στην βελτίωση της εμπειρίας και της διαχειρισιμότητας, μπορούν, ως παράπλευρη συνέπεια, να οδηγήσουν σε αύξηση του latency ή της κατανάλωσης πόρων. Το φαινόμενο αυτό αποδίδεται στην πολυπλοκότητα των αλληλεπιδράσεων μεταξύ μικροϋπηρεσιών και την έλλειψη πλήρους visibility σε runtime επίπεδο \cite{kleppmanndda}.

Τα περισσότερα κατανεμημένα συστήματα τη σημερινή εποχή βασίζονται στο \textit{Apache Kafka} για την αποστολή μηνυμάτων και τη ροή δεδομένων μεταξύ διαφορετικών υπηρεσιών και επιπέδων επεξεργασίας. Το Kafka έχει εξελιχθεί σε de facto πρότυπο για την υλοποίηση ανθεκτικών και επεκτάσιμων messaging υποδομών, χάρη στη δυνατότητά του να αποθηκεύει τα μηνύματα σε μορφή καταγραφής (\textit{log-based}) και να υποστηρίζει semantics παράδοσης τύπου \textit{at-least-once} και \textit{exactly-once}. Σε αυτό το πλαίσιο, νεότερα συστήματα όπως το \textit{NATS}, προτείνουν μια εναλλακτική προσέγγιση εστιασμένη στη χαμηλή καθυστέρηση, την απλότητα και την ευκολία ενσωμάτωσης σε \textit{cloud-native} υποδομές. Το NATS ακολουθεί \textit{publish-subscribe} μοντέλο επικοινωνίας, με υποστήριξη για \textit{persistence} και \textit{streaming} μέσω του υποσυστήματος JetStream. Σε αντίθεση με το Kafka, το οποίο έχει σχεδιαστεί για μαζική επεξεργασία δεδομένων και υψηλή αντοχή σε failure scenarios, το NATS προσανατολίζεται κυρίως σε περιπτώσεις όπου η ταχύτητα και η απλότητα ενσωμάτωσης υπερέχουν της μακροχόνιας αποθήκευσης. Περαιτέρω πλεονέκτημα του NATS έιναι η μη εξάρτηση του από εξωτερικά συστήματα όπως το Zookeeper.

Η αποθήκευση μεγάλου όγκου δεδομένων σε κατανεμημένα συστήματα βασίζεται συχνά σε βάσεις δεδομένων τύπου \textit{wide-column store}, με το \textit{Apache Cassandra} να αποτελεί ένα από τα πιο διαδεδομένα και ώριμα συστήματα σε αυτόν τον χώρο. Το Cassandra ακολουθεί το μοντέλο της \textit{eventual consistency}, υποστηρίζει κατανεμημένη αποθήκευση με replication και partitioning, και έχει σχεδιαστεί για \textit{write-heavy} εφαρμογές. Η αρχιτεκτονική του Cassandra βασίζεται στο Java Virtual Machine (JVM), γεγονός που επιτρέπει εύκολη συντήρηση και ευρεία διαθεσιμότητα εργαλείων, αλλά ταυτόχρονα εισάγει περιορισμούς σχετικούς με τη διαχείριση μνήμης, το garbage collection (GC), και την απόδοση σε περιβάλλοντα υψηλού throughput. Ειδικά σε σενάρια όπου το latency αποτελεί κρίσιμο παράγοντα, η μη προβλεψιμότητα της απόδοσης λόγω του GC μπορεί να οδηγήσει σε ανεπιθύμητες χρονικές αποκλίσεις. Για την αντιμετώπιση τέτοιων περιορισμών, αναπτύχθηκε το \textit{ScyllaDB}, μια drop-in εναλλακτική του Cassandra, γραμμένη εξολοκλήρου σε C++ και βασισμένη στο \textit{Seastar framework} — ένα μοντέλο πλήρως ασύγχρονου ,\textit{shared-nothing} event-driven προγραμματισμού. Το ScyllaDB επιτυγχάνει καλύτερη αξιοποίηση των υπολογιστικών πόρων, προσφέροντας αυξημένο throughput και σταθερότερα latency χαρακτηριστικά \cite{scyllainaction}.

Σε ότι αφορά την ανάγνωση των αποθηκευμένων δεδομένων αυτού του όγκου, αλλά και δεδομένης της συνεχόμενης άφιξης τους, τα προσωρινά δεδομένα σε μνήμη (\textit{in-memory caching}) αποτελούν κρίσιμο μηχανισμό βελτιστοποίησης της απόδοσης. Δυο από τις πιο διαδεδομένες τεχνολογίες σε αυτό το πεδίο είναι τα συστήματα \textit{Memcached} και \textit{Redis}, τα οποία προσφέρουν μηχανισμούς προσωρινής αποθήκευσης κλειδιών και τιμών, με στόχο τη μείωση του χρόνου αλλά και των προσπαθειών προσπέλασης σε βάσεις δεδομένων. Παράλληλα, τα τελευταία χρόνια έχει εμφανιστεί το \textit{DragonflyDB}, μια νέα υλοποίηση caching, βασισμένη στο Redis (και πλήρως συμβατή με το API και τους drivers του). Το Dragonfly αξιοποιεί τεχνικές όπως \textit{shared-nothing concurrency} και βελτιστοποιήσεις στον τρόπο διαχείρισης μνήμης και threads (thread-per-core), προσφέροντας καλύτερη απόδοση σε καλύτερο κόστος υποδομών, λόγω της αποφυγής εσωτερικών συμφορήσεων (\textit{bottlenecks}) και της βέλτιστης αξιοποίησης των διαθέσιμων υπολογιστικών πόρων. Ενώ το Redis και το Memcached εξακολουθούν να είναι οι πλέον αξιόπιστες επιλογές σε περιβάλλοντα παραγωγής, το DragonflyDB εξετάζεται όλο και περισσότερο από την ερευνητική και τεχνική κοινότητα λόγω των επιδόσεων του, της απλότητας εγκατάστασης (χωρίς εξωτερικά dependencies) και της δυνατότητας κάθετου scaling.

Η επιλογή των παραπάνω τεχνολογιών δεν γίνεται αποκομμένα, αλλά εντάσσεται σε μία ευρύτερη προσπάθεια υλοποίησης υποδομών που να μπορούν να ανταποκρίνονται δυναμικά σε συνθήκες αβεβαιότητας, αστοχίας ή φόρτου. Η ανάγκη για παρατηρησιμότητα, αποδοτική διαχείριση προσωρινών δεδομένων και σταθερή ροή μηνυμάτων σε περιβάλλοντα με κατανεμημένη υπολογιστική λογική, καθιστά την πλατφόρμα \textit{Kubernetes} απαραίτητο δομικό στοιχείο, καθώς επιτρέπει την υλοποίηση μηχανισμών αυτόματης αποκατάστασης, \textit{autoscaling}, και επιτήρησης σε επίπεδο υπηρεσίας.

Κατ’ επέκταση, η ερευνητική συνεισφορά εστιάζει όχι μόνο στην επιλογή επιμέρους τεχνολογιών, αλλά κυρίως στη συνεκτική και δυναμική ενορχήστρωσή τους, με στόχο την επίτευξη αυτονομίας και διαχειρισιμότητας σε cloud-native κατανεμημένα περιβάλλοντα υψηλής πολυπλοκότητας.
