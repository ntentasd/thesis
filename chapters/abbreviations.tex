\section*{Συντομογραφίες}

\markboth{ΑΚΡΩΝΥΜΙΑ}{ΑΚΡΩΝΥΜΙΑ}
\noindent
\textbf{Ο παρακάτω πίνακας περιλαμβάνει μία συνοπτική λίστα με τα τεχνικά ακρωνύμια και συντομογραφίες που χρησιμοποιούνται σε αυτό το έγγραφο.}
\vspace{0.5em}

\begin{longtable}{|l|p{11.5cm}|}
    \hline
    \textbf{Ακρωνύμιο} & \textbf{Περιγραφή} \\
    \hline
    API & Application Programming Interface \\
    \hline
    CoAP & Constrained Application Protocol \\
    \hline
    CPS & Cyber-Physical System \\
    \hline
    CPU & Central Processing Unit \\
    \hline
    DSL & Domain Specific Language \\
    \hline
    EFK & ElasticSearch, FluentBit, Kibana \\
    \hline
    ELK & ElasticSearch, Logstash, Kibana \\
    \hline
    GC & Garbage Collection \\
    \hline
    gRPC & gRPC Remote Procedure Call \\
    \hline
    I/O & Input/Output \\
    \hline
    IoT & Internet of Things \\
    \hline
    JSON & JavaScript Object Notation \\
    \hline
    JVM & Java Virtual Machine \\
    \hline
    LRU & Least Recently Used \\
    \hline
    MQTT & Message Queuing Telemetry Transport \\
    \hline
    MVP & Minimum Viable Product \\
    \hline
    OTLP & Open TeLemetry Protocol \\
    \hline
    P95 & 95th Percentile \\
    \hline
    QoS & Quality of Service \\
    \hline
    RBAC & Role-Based Access Control \\
    \hline
    RPS & Requests Per Second \\
    \hline
    SLI & Service Level Indicator \\
    \hline
    TTL & Time To Live \\
    \hline
    UI & User Interface \\
    \hline
    \caption{Κατάλογος Ακρωνυμίων και Συντομογραφιών} \label{tab:abbreviations} \\
\end{longtable}
