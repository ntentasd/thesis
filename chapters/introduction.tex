\chapter{Εισαγωγή}

\section{Πλαισίωση του Προβλήματος}

% Στην αυγή της τέταρτης βιομηχανικής επανάστασης, ο πρωτογενής τομέας διέρχεται
% μια φάση ριζικού μετασχηματισμού, όπου η πληροφορία μετατρέπεται στο
% πολυτιμότερο αγροτικό εφόδιο. Η ενσωμάτωση του Διαδικτύου των Πραγμάτων
% (Internet of Things - IoT) και των Κυβερνοφυσικών Συστημάτων (CPS) στην
% παραγωγική διαδικασία υπόσχεται τη βελτιστοποίηση των πόρων και την ενίσχυση
% της αποδοτικότητας. Ωστόσο, η αυξανόμενη εξάρτηση από τις ψηφιακές υποδομές
% αναδεικνύει μια νέα, κρίσιμη πρόκληση: την εύθραυστη φύση της πολυπλοκότητας.
%
% Σε περιβάλλοντα όπως η γεωργία ακριβείας, η αποτυχία ενός μεμονωμένου κόμβου ή
% η καθυστέρηση στη μεταφορά δεδομένων δεν αποτελεί απλώς ένα τεχνικό σφάλμα,
% αλλά μια απειλή για την επιχειρησιακή συνέχεια και την επισιτιστική ασφάλεια. Η
% ανάγκη, επομένως, μετατοπίζεται από την απλή συλλογή δεδομένων στην οικοδόμηση
% ανθεκτικών (resilient) αρχιτεκτονικών, ικανών να αυτοθεραπεύονται και να
% προσαρμόζονται σε δυναμικές συνθήκες.

Στην αυγή της τέταρτης βιομηχανικής επανάστασης, ο πρωτογενής τομέας
μετασχηματίζεται σε ένα πεδίο έντασης δεδομένων. Η ενσωμάτωση του Διαδικτύου
των Πραγμάτων (IoT) και των Κυβερνοφυσικών Συστημάτων (CPS) υπόσχεται τη
βελτιστοποίηση των πόρων, ωστόσο δημιουργεί μια κρίσιμη τεχνική αντίφαση:
\textbf{την εξάρτηση ζωτικών λειτουργιών από εγγενώς ασταθείς ψηφιακές
υποδομές}.

Το πρόβλημα στο οποίο εστιάζει η παρούσα εργασία εντοπίζεται στην
\textbf{αρχιτεκτονική ευθραυστότητα} των υπαρχουσών πλατφορμών IoT. Ενώ η
συλλογή δεδομένων έχει επιλυθεί σε μεγάλο βαθμό, οι σύγχρονες υποδομές υστερούν
στους εξής τρεις άξονες:

\begin{enumerate}
    \item \textbf{Συμφόρηση και Καθυστέρηση (Bottlenecks):} Η εκθετική αύξηση
	    του όγκου των δεδομένων από αισθητήρες πεδίου προκαλεί
	    καθυστερήσεις στην επεξεργασία σε πραγματικό χρόνο, καθιστώντας τα
	    συστήματα ανίκανα να ανταποκριθούν σε κρίσιμα γεγονότα (π.χ. ανάγκη
	    άμεσης άρδευσης).
    \item \textbf{Έλλειψη Εσωτερικής Ορατότητας (Observability Gap):} Τα
	    περισσότερα συστήματα λειτουργούν ως "μαύρα κουτιά". Όταν
	    παρουσιάζεται μια δυσλειτουργία, η διάγνωση της αιτίας είναι
	    εξαιρετικά δύσκολη, οδηγώντας σε παρατεταμένους χρόνους εκτός
	    λειτουργίας (downtime).
    \item \textbf{Αδυναμία Αυτόνομης Ανάκαμψης:} Σε απομακρυσμένες αγροτικές
	    περιοχές, η φυσική παρέμβαση για την επιδιόρθωση σφαλμάτων
	    λογισμικού είναι κοστοβόρα ή αδύνατη. Η έλλειψη μηχανισμών
	    αυτοΐασης (self-healing) σημαίνει ότι ένα μεμονωμένο σφάλμα μπορεί
	    να θέσει εκτός λειτουργίας ολόκληρο το δίκτυο παρακολούθησης.
\end{enumerate}

Συνεπώς, το κεντρικό πρόβλημα δεν είναι η έλλειψη δεδομένων, αλλά η
\textbf{έλλειψη ανθεκτικότητας και παρατηρησιμότητας} των συστημάτων που τα
διαχειρίζονται, γεγονός που θέτει σε κίνδυνο την επιχειρησιακή συνέχεια και την
επισιτιστική ασφάλεια.

\section{Κίνητρο και Ερευνητικά Ερωτήματα}

Η παρούσα εργασία ελλοχεύει από την παρατήρηση ότι οι υπάρχουσες \textit{IoT}
πλατφόρμες συχνά θυσιάζουν την παρατηρησιμότητα (observability) και την
ανθεκτικότητα στον βωμό της ταχείας ανάπτυξης. Ιδιαίτερα για την ελληνική
πραγματικότητα, όπου ο αγροτικός τομέας απαιτεί λύσεις χαμηλού κόστους αλλά
υψηλής αξιοπιστίας, η έρευνα γύρω από βελτιστοποιημένες \textit{cloud-native}
αρχιτεκτονικές καθίσταται επιτακτική.

Τα κεντρικά ερωτήματα που εξετάζονται είναι:

\begin{itemize}
	\item Μπορεί μια κατανεμημένη αρχιτεκτονική να διασφαλίσει τη
		συνέχεια της ροής δεδομένων υπό συνθήκες υψηλού φόρτου ή
		αστοχίας;
	\item Με ποιον τρόπο η ενσωμάτωση μηχανισμών \textit{caching} και
		\textit{self-healing} επηρεάζει το \textit{latency} και την
		ολική απόδοση ενός συστήματος;
	\item Ποιοι είναι οι βέλτιστοι τρόποι διαχείρισης της ετερογένειας των
		δεδομένων σε κρίσιμες υποδομές, όπως για παράδειγμα δίκτυα
		ύδρευσης και ενέργειας;
\end{itemize}

\section{Σκοπός και Αντικείμενο της Εργασίας}

Η παρούσα διπλωματική εργασία εστιάζει στην ανάπτυξη και βελτιστοποίηση μιας
ολοκληρωμένης αρχιτεκτονικής για την παροχή προηγμένων υπηρεσιών παρατήρησης
και λήψης αποφάσεων στον τομέα της έξυπνης γεωργίας και της επισιτιστικής
ασφάλειας. Η πλατφόρμα φιλοξενεί την κρίσιμη υποδομή "IoT Observability", η
οποία λειτουργεί ως ο κεντρικός πυλώνας συλλογής και επεξεργασίας δεδομένων από
ένα ευρύ δίκτυο IoT και Edge συσκευών.

Κεντρικός σκοπός της εργασίας δεν είναι μόνο η απλή παρακολούθηση της υποδομής,
αλλά η ολιστική αρχιτεκτονική αναβάθμιση μέσω της ενσωμάτωσης τριών
αλληλένδετων μηχανισμών:

\begin{itemize}
	\item \textbf{Caching:} Για τη βελτιστοποίηση της απόδοσης και την
		εξάλειψη των σημείων συμφόρησης κατά την ανάκτηση σύνθετων
		συναθροίσεων.
	\item \textbf{Παρατηρησιμότητα συστήματος:} Για τη δημιουργία μιας
		διαφανούς υποδομής που επιτρέπει τη βαθιά διάγνωση της
		κατάστασης του συστήματος σε πραγματικό χρόνο.
	\item \textbf{Αυτοΐαση:} Για την παροχή προληπτικής ικανότητας
		ανάκαμψης από αστοχίες, διασφαλίζοντας την επιχειρησιακή
		συνέχεια.
\end{itemize}

Το αντικείμενο της μελέτης επικεντρώνεται στη μεταστοιχείωση της πλατφόρμας από
μια στατική υποδομή σε ένα ανθεκτικό Κυβερνοφυσικό Σύστημα (CPS), ικανό να
υποστηρίξει κρίσιμες εφαρμογές γεωργίας ακριβείας υπό συνθήκες υψηλού φόρτου
και μεταβλητότητας.

\section{Συνεισφορά της Εργασίας}

Η συνεισφορά της παρούσας διπλωματικής εργασίας επικεντρώνεται στον σχεδιασμό,
την υλοποίηση και την αξιολόγηση ενός διευρυμένου αρχιτεκτονικού πλαισίου για
συστήματα μεγάλης κλίμακας. Συγκεκριμένα:

\begin{itemize}
	\item \textbf{Αρχιτεκτονική Προσέγγιση:} Προτείνεται ένα πολυεπίπεδο
		μοντέλο που ενσωματώνει κατανεμημένη προσωρινή αποθήκευση
		(distributed caching) για τη μείωση της συμφόρησης.
	\item \textbf{Μηχανισμοί Ανθεκτικότητας:} Υλοποιούνται πρωτόκολλα
		αυτόματης διάγνωσης και ανάκαμψης, βασισμένα στο πλαίσιο
		\textit{R4} (Redundancy, Robustness, Resourcefulness,
		Rapidity).
	\item \textbf{Αξιολόγηση σε Πραγματικές Συνθήκες:} Αναλύεται η επίδραση
		των προτεινόμενων λύσεων στην απόδοση του συστήματος,
		αναδεικνύοντας τις αναγκαίες συμβιβαστικές λύσεις (trade-offs)
		μεταξύ πολυπλοκότητας και απόδοσης.
	\item \textbf{Ενίσχυση Παρατηρησιμότητας:} Προτείνεται μια
		\textit{end-to-end} λύση/πλατφόρμα υποστήριξης της κύριας
		υποδομής, με βασικό στόχο τη συνεχή διάγνωση της κατάστασης
		(υγείας) του συστήματος, μέσω της συλλογής μετρικών όλων των
		ειδών.
\end{itemize}

\section{Δομή της Εργασίας}

Η παρούσα διπλωματική εργασία διαρθρώνεται σε επτά κεφάλαια, ακολουθώντας μια
λογική πορεία από τη θεωρητική θεμελίωση στην τεχνική υλοποίηση και την
πειραματική επαλήθευση:

\begin{itemize}
	\item \textbf{Κεφάλαιο 2 - Επισκόπηση της Ερευνητικής Περιοχής:}
		Παρουσιάζεται το πλαίσιο των Κυβερνοφυσικών Συστημάτων (CPS)
		στη σύγχρονη γεωργία και αναλύονται οι έννοιες της
		ανθεκτικότητας και της επισιτιστικής ασφάλειας.
	\item \textbf{Κεφάλαιο 3 - Θεωρητικό Υπόβαθρο:} Εξετάζονται οι
		αρχιτεκτονικές κατανεμημένων συστημάτων, τα μοντέλα
		επεξεργασίας ροών δεδομένων, καθώς και οι αρχές του
		\textit{cloud-native computing} και των μοντέλων συνέπειας.
	\item \textbf{Κεφάλαιο 4 - Εργαλεία:} Γίνεται αναλυτική παρουσίαση των
		τεχνολογιών που χρησιμοποιήθηκαν, συγκεκριμένα για τη ροή
		δεδομένων, για την αποθήκευση, καθώς για το επίπεδο προσωρινής
		αποθήκευσης/ταχείας προσπέλασης.
	\item \textbf{Κεφάλαιο 5 - Υλοποιήσεις:} Αποτελεί τον κεντρικό πυλώνα
		της εργασίας, όπου περιγράφεται η αρχιτεκτονική υποδομή, η
		μετακίνηση των δεδομένων μεταξύ των υποσυστημάτων, και η
		ενσωμάτωση των επιπέδων παρατηρησιμότητας και ανθεκτικότητας.
	\item \textbf{Κεφάλαιο 6 - Πειράματα:} Παρουσιάζεται η μεθοδολογία και
		η εκτέλεση των δοκιμών. Περιλαμβάνει τη σύγκριση απόδοσης των
		πειραμάτων και τη μελέτη της επίδρασης της χρονικής πυκνότητας
		των δεδομένων στην ευστάθεια του συστήματος.
	\item \textbf{Κεφάλαιο 7 - Συμπεράσματα:} Ανακεφαλαιώνονται τα ευρήματα
		της έρευνας, αξιολογείται η επίτευξη των στόχων και
		προτείνονται κατευθύνσεις για μελλοντικές επεκτάσεις του
		συστήματος.
\end{itemize}
