\chapter{Επισκόπηση της Ερευνητικής Περιοχής}

Στη σύγχρονη γεωργία, ένας δυσλειτουργικός αισθητήρας που περνά απαρατήρητος για ώρες μπορεί να οδηγήσει σε αποτυχημένο πότισμα ή καταστροφή καλλιεργειών. Αυτό το ενιαίο σημείο αποτυχίας αναδεικνύει την ανάγκη για αυτόνομα, ανθεκτικά και παρατηρήσιμα συστήματα σε κλίμακα. Η ανάγκη αυτή εναρμονίζεται με τον οδικό χάρτη (Roadmap) προς ανθεκτικά Κυβερνοφυσικά Συστήματα (CPS) που περιγράφεται από τους Ratasich et al. \cite{iotcps}, οι οποίοι δίνουν έμφαση στην ανίχνευση ανωμαλιών κατά τη διάρκεια λειτουργίας, στην απομόνωση σφαλμάτων και στην αυτοΐαση σε δυναμικά Internet of Things (IoT) περιβάλλοντα.

Οι συσκευές IoT παράγουν, σε πραγματικό χρόνο, μεγάλου όγκου δεδομένα, είτε σε μορφή χρονοσειράς, είτε ως ιστορικά δεδομένα \cite{rtiotevents}. Επομένως, η κλίμακα αυτή αποτελεί βασική αιτία εμφάνισης προβλημάτων και αστοχιών. Ωστόσο, για να εξασφαλιστεί η ομαλή λειτουργία των κρίσιμων υποδομών ενός νευραλγικού τομέα, όπως η γεωργία, οφείλουμε ως μηχανικοί να παρέχουμε λύσεις για τον παραγωγό και κατά συνέπεια για τον πολίτη, που εξασφαλίζουν τόσο την ποιότητα, όσο και την αναμενόμενη ποσότητα των αγροτικών προιόντων. Σε χώρες όπως η Ελλάδα, όπου η οικονομία και η αυτονομία, έγκεινται σε μεγάλο βαθμό στη γεωργική παραγωγή, η αξιοποίηση των τεχνολογιών πραγματικού χρόνου αποτελεί κρίσιμο παράγοντα για τη διασφάλιση της βιωσιμότητας και της αποδοτικότητας του πρωτογενούς τομέα. Η ανάπτυξη καινοτόμων υποδομών που ενσωματώνουν σύγχρονες τεχνολογίες όπως το \textit{Apache Kafka} για επεξεργασία σε πραγματικό χρόνο προσφέρει τη δυνατότητα συνεχούς παρακολούθησης, έγκαιρης παρέμβασης και λήψης αποφάσεων βασισμένων σε δεδομένα. Έτσι, ενισχύεται η ανθεκτικότητα του αγροδιατροφικού τομέα απέναντι στις προκλήσεις της εποχής, όπως η κλιματική αλλαγή, οι μεταβολές στη ζήτηση και η ανάγκη για βιώσιμη διαχείριση πόρων.

Σε κρίσιμες εφαρμογές, όπως η αγροδιατροφή, η διατάραξη στη ροή δεδομένων ή στη λήψη των αποφάσεων μπορεί να οδηγήσει σε σπατάλη τροφίμων, οικονομική απώλεια ή επισιτιστική ανασφάλεια. Σύμφωνα με τους Callo και Mansouri \cite{foodsec}, η ανθεκτικότητα των παγκόσμιων δικτύων διανομής τροφίμων εξαρτάται από την δυνατότητα των πληροφοριακών συστημάτων να ανταπεξέρχονται σε γεωπολιτικές ή υγειονομικές κρίσεις μέσω μηχανισμών ευελιξίας και προσαρμογής. Αυτό σημαίνει πως σε ότι αφορά σε αντίστοιχα συστήματα, στα οποία βασίζεται ο πλυθησμός (και η οικονομία) μιας χώρας, η αρχιτεκτονική που χρησιμοποιείται θα πρέπει να είναι άρτια, μελετημένη, αλλά και να εξασφαλίζει την ομαλή και συνεχή λειτουργία τους.

Οι σύγχρονες cloud-native τεχνολογίες παρέχουν τα εργαλεία και τα πρότυπα για την υλοποίηση τέτοιων χαρακτηριστικών, και την πρόληψη των προαναφερθέντων επιπτώσεων. Μέσω συνεχόμενης παρατήρησης ορισμένων μετρικών (\textit{Observability}), αλλά και ειδοποιήσεων σε κρίσιμες περιπτώσεις (\textit{Alerting}), μπορεί είτε να επανορθωθεί η ομαλή λειτουργία του συστήματος αυτόματα, είτε να ενημερωθεί ο αρμόδιος για την επανόρθωση χειριστής. Η ενσωμάτωση εργαλείων όπως το Prometheus σε cloud-native μικροϋπηρεσίες έχει ήδη αποδειχθεί αποτελεσματική σε εφαρμογές πραγματικού χρόνου, παρέχοντας κρίσιμη τηλεμετρία και αυτόματες ενέργειες ανάδρασης \cite{iotmonitoring}.

Βάσει της μελέτης των Sharma et al. \cite{iotagriculture}, αναδεικνύεται η ολοένα αυξανόμενη σημασία της χρήσης IoT τεχνολογιών στον τομέα της γεωργίας ακριβείας, όπου η αξιοποίηση αισθητήρων για την παρακολούθηση παραμέτρων όπως η υγρασία, η θερμοκρασία, η αγωγιμότητα και τα επίπεδα θρεπτικών συστατικών του εδάφους (NPK), επιτρέπει την ακριβή λήψη αποφάσεων σε πραγματικό χρόνο. Η προσέγγιση αυτή ευθυγραμμίζεται με την ανάγκη για ευέλικτες, αποκεντρωμένες υποδομές λήψης αποφάσεων, οι οποίες υποστηρίζονται από μηχανισμούς edge analytics και cloud ενορχήστρωσης. Στο πλαίσιο αυτό, η μελέτη των Akhtar et al. \cite{edgeagriculture} τονίζει τη σημασία της ενσωμάτωσης του edge computing για την επεξεργασία δεδομένων από αισθητήρες σε πραγματικό χρόνο, επιτρέποντας την αξιολόγηση του εδάφους και την παρακολούθηση ρύπων με μεγαλύτερη αποτελεσματικότητα. Υποστηρίζει, επιπλέον, πως η ενσωμάτωση τεχνικών μηχανικής μάθησης στην αλυσίδα επεξεργασίας των δεδομένων, όπως η αυτόματη πρόβλεψη της καταλληλότητας του εδάφους για συγκεκριμένες καλλιέργειες, ενισχύει την αξία των IoT συστημάτων και απαιτεί αρχιτεκτονική σχεδίαση που υποστηρίζει δυναμική ανάλυση και διαλειτουργικότητα. Τα ευρήματα αυτά ενισχύουν την άποψη ότι η αρχιτεκτονική ενός ανθεκτικού IoT συστήματος στον αγροδιατροφικό τομέα πρέπει να υποστηρίζει συνεχές monitoring, on-device προεπεξεργασία και ασφαλή, επεκτάσιμη μετάδοση δεδομένων, προκειμένου να επιτευχθεί πλήρης αυτοματοποίηση και ευστάθεια λειτουργίας σε ετερογενή, διασυνδεδεμένα περιβάλλοντα πεδίου.

Με στόχο, λοιπόν, την αρχιτεκτονική ενίσχυση του συστήματος \textit{Nostradamus} προτείνεται, ως απόρροια των παραπάνω μελετών, η ενσωμάτωση μηχανισμών caching, observability και self-healing. Παρότι υπάρχουν επιμέρους τεχνολογίες που προσφέρουν τέτοιες δυνατότητες, η υιοθέτησή τους σε πραγματικές πολυεπίπεδες IoT πλατφόρμες συνοδεύεται από σημαντικές προκλήσεις. Στα κατανεμημένα συστήματα, η προσθήκη μιας νέας υπηρεσίας ή λειτουργικότητας συνοδεύεται συχνά από μη προβλέψιμες επιπτώσεις στην απόδοση του συστήματος, όπως αναφέρει επανειλημμένα ο Kleppmann \cite{kleppmanndda}. Αν και οι μηχανισμοί όπως το caching και το observability αποσκοπούν στην βελτίωση της εμπειρίας και της διαχειρισιμότητας, μπορούν, ως παράπλευρη συνέπεια, να οδηγήσουν σε αύξηση του latency ή της κατανάλωσης πόρων. Το φαινόμενο αυτό αποδίδεται στην πολυπλοκότητα των αλληλεπιδράσεων μεταξύ μικροϋπηρεσιών και την έλλειψη πλήρους visibility σε runtime επίπεδο.

Κατ’ επέκταση, η ερευνητική συνεισφορά εστιάζει όχι μόνο στην επιλογή σχετικών τεχνολογιών, αλλά κυρίως στη συνεκτική και δυναμική ενορχήστρωσή τους, με στόχο την επίτευξη αυτονομίας και διαχειρισιμότητας σε cloud-native κατανεμημένα περιβάλλοντα υψηλής πολυπλοκότητας, για την υποστήριξη αυτού του συστήματος πραγματικού χρόνου στο χώρο του \textit{food security}.
