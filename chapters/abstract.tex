\chapter*{Περίληψη}
\addcontentsline{toc}{chapter}{Περίληψη}

Η παρούσα διπλωματική εργασία εστιάζει στον σχεδιασμό, την υλοποίηση και την
πειραματική αξιολόγηση μιας ανθεκτικής, cloud-native υποδομής δεδομένων IoT για
πλατφόρμες γεωργίας ακριβείας. Με δεδομένη την κρισιμότητα των Κυβερνοφυσικών
Συστημάτων (CPS) στον αγροδιατροφικό τομέα, η έρευνα επικεντρώνεται στην
ενσωμάτωση τριών κρίσιμων πυλώνων: της προσωρινής αποθήκευσης (caching), της
παρατηρησιμότητας (observability) και των μηχανισμών αυτοΐασης (self-healing).

Η αρχιτεκτονική της πλατφόρμας βασίζεται στο Kubernetes και αξιοποιεί ένα
σύγχρονο data stack αποτελούμενο από Apache Kafka για τη διαχείριση ροών,
ScyllaDB για την αποθήκευση χρονοσειρών και Arroyo για την επεξεργασία
streaming δεδομένων. Η παρατηρησιμότητα του συστήματος ενισχύθηκε μέσω ενός
πλήρους stack (Prometheus, Grafana, Loki, Tempo), το οποίο επέτρεψε την
ιχνηλάτηση (tracing) των αιτημάτων από το API layer έως το επίπεδο αποθήκευσης,
διευκολύνοντας την ανάλυση βαθύτερης αιτίας (Root Cause Analysis).

Η πειραματική διαδικασία επικεντρώθηκε στην αξιολόγηση εναλλάξιμων cache
drivers (Valkey, Memcached) υπό ρεαλιστικό φόρτο 350 RPS. Τα αποτελέσματα
ανέδειξαν ένα κρίσιμο trade-off: ενώ το Memcached πέτυχε χαμηλότερη μέση
καθυστέρηση (P95 read στα 486 μs), το Valkey επέδειξε ανώτερη σταθερότητα στα
υψηλά percentiles (P99), διατηρώντας το tail latency κάτω από τα 5.6 ms ακόμη
και σε συνθήκες υποβέλτιστης πυκνότητας δεδομένων (data granularity). Επιπλέον,
διαπιστώθηκε ότι η αυξημένη συχνότητα μετρήσεων χωρίς αντίστοιχη πληροφοριακή
αξία μπορεί να επιβαρύνει το cache layer έως και 56\%.

Τα συμπεράσματα της εργασίας τεκμηριώνουν ότι η συνδυαστική χρήση stateful
επεξεργασίας ροών και προηγμένης παρατηρησιμότητας επιτρέπει την ανάπτυξη
συστημάτων που όχι μόνο αποδίδουν σε υψηλή κλίμακα, αλλά διαθέτουν την
απαιτούμενη διαγνωστική ικανότητα για την υποστήριξη στρατηγικών self-healing
και multi-cloud ανάπτυξης.

\chapter*{Abstract}
\addcontentsline{toc}{chapter}{Abstract}

This master thesis investigates the design, implementation, and experimental
evaluation of a resilient, cloud-native IoT data infrastructure for the
precision agriculture platforms. Given the critical nature of Cyber-Physical
Systems (CPS) in the agrifood sector, this research focuses on the integration
of three key pillars: distributed caching, full-stack observability, and
self-healing mechanisms.

The platform architecture is built on Kubernetes, utilizing a modern data stack
comprising Apache Kafka for stream management, ScyllaDB for time-series
storage, and Arroyo for real-time stream processing. System-level observability
was enhanced through a comprehensive stack (Prometheus, Grafana, Loki, Tempo),
enabling end-to-end request tracing and facilitating rapid Root Cause Analysis
(RCA) across the infrastructure.

Experimental evaluation focused on the comparative performance of
interchangeable cache drivers (Valkey, Memcached) under a realistic load of 350
RPS. The results revealed a significant architectural trade-off: while
Memcached provided lower average latency (P95 read at 486 μs), Valkey
demonstrated superior stability in higher percentiles (P99), maintaining tail
latency below 5.6 ms even under suboptimal data granularity conditions.
Furthermore, it was observed that excessive measurement frequency without
proportional information value could increase cache write latency by up to
56\%.

The findings of this thesis demonstrate that the synergy of stateful stream
processing and advanced observability enables the development of IoT
infrastructures that are not only performant at scale but also possess the
diagnostic depth required to support autonomous recovery and multi-cloud
resilience strategies.
