\chapter{Εργαλεία}

Τα Κυβερνο-Φυσικά Συστήματα (CPS) είναι συστήματα που αποτελούνται από ένα
φυσικό στοιχείο το οποίο ελέγχεται ή παρακολουθείται από ένα κυβερνητικό
(cyber) στοιχείο, έναν αλγόριθμο βασισμένο σε υπολογιστή. Με στόχο να
μετασχηματίσουν τον τρόπο με τον οποίο οι άνθρωποι αλληλεπιδρούν με τα μηχανικά
συστήματα, τα νέα έξυπνα CPS οδηγούν την καινοτομία σε διάφορους τομείς,
βασικός εκ των οποίων αποτελεί η γεωργία \cite{cps}. Η αρχιτεκτονική των CPS
βασίζεται σε τρία βασικά επίπεδα: το φυσικό επίπεδο (physical layer), όπου
καταγράφονται και παράγονται τα δεδομένα μέσω αισθητήρων· το επίπεδο δικτύου
(network layer), που εξασφαλίζει τη μεταφορά των δεδομένων· και το κυβερνητικό
ή υπολογιστικό επίπεδο (cyber layer), όπου λαμβάνονται αποφάσεις βάσει των
εισερχόμενων δεδομένων. Η συνύπαρξη αυτών των στρωμάτων σε ένα κοινό σύστημα
καθιστά τα CPS ιδιαίτερα κατάλληλα για εφαρμογές που απαιτούν χαμηλή
καθυστέρηση, αξιοπιστία και αυτονομία. Όπως προαναφέρθηκε, στο πεδίο της
γεωργίας ακριβείας, τα CPS διαδραματίζουν καθοριστικό ρόλο, καθώς συνδυάζουν
αισθητήρες πεδίου, μηχανισμούς ελέγχου άρδευσης, και αλγορίθμους πρόβλεψης
βασισμένους σε δεδομένα για να εξασφαλίσουν βέλτιστες συνθήκες καλλιέργειας. Οι
τεχνολογίες αυτές επιτρέπουν τη δυναμική λήψη αποφάσεων, μειώνουν τις απώλειες
και αυξάνουν την αποδοτικότητα σε όλα τα στάδια της παραγωγής.

\section{Kafka}

Για να επιτευχθεί όμως η πλήρης δυναμική των CPS, απαιτούνται ισχυρές υποδομές
διασύνδεσης και διαχείρισης δεδομένων. Εδώ εντάσσεται η ανάγκη για αξιόπιστες
messaging πλατφόρμες, όπως το Apache Kafka, που διασφαλίζουν την συνεχή και
αξιόπιστη ροή πληροφοριών ανάμεσα στα υποσυστήματα ενός CPS. Το Apache Kafka
αποτελεί ένα - πλέον - de facto πρότυπο για την υλοποίηση τέτοιων messaging
υποδομών, χάρη στη δυνατότητα του να αποθηκεύει τα μηνύματα σε μορφή καταγραφής
(log-based) \cite{kafkabdd}. Η αρχιτεκτονική του Kafka ενδείκνυται ιδιαίτερα
για περιβάλλοντα που απαιτούν real ή near-real time επεξεργασία γεγονότων,
καθώς παρέχει υψηλή διαθεσιμότητα, εγγυημένη διανομή μηνυμάτων και δυνατότητα
οριζόντιας κλιμάκωσης. Συγκριτικές μελέτες \cite{rtkafka} επιβεβαιώνουν ότι το
Kafka παρουσιάζει σημαντικό πλεονέκτημα σε όρους throughput και fault tolerance
σε σχέση με άλλες προσεγγίσεις, γεγονός που το καθιστά κατάλληλο για
απαιτητικές IoT εφαρμογές με αυξημένο όγκο και ταχύτητα δεδομένων.

Σε ένα σύστημα όπως το \textit{Nostradamus}, όπου η ροή των δεδομένων είναι
αδιάκοπη και εξελίσσεται σε πραγματικό χρόνο, η ταχεία εισαγωγή και εξαγωγή των
δεδομένων εκτιμάται, τόσο για λόγους απόδοσης όσο και για την ελαχιστοποίηση
της κατανάλωσης υπολογιστικών πόρων. Δοθέντος ενός συνόλου αισθητήρων
τοποθετημένων σε έναν αγρό, οι οποίοι παράγουν συνεχώς δεδομένα, ο κεντρικός
broker πρέπει να τα λαμβάνει ορθά και εντός λογικών χρονικών πλαισίων, ενώ
ταυτόχρονα να τα επεξεργάζεται χωρίς να καταπονεί το συνολικό σύστημα. Συνεπώς,
πρέπει να ληφθούν υπόψη τόσο η ποιότητα υπηρεσίας (Quality of Service - QoS)
της messaging υποδομής όσο και οι εγγυήσεις παράδοσης που αυτή προσφέρει. Στη
συγκεκριμένη εφαρμογή, η απώλεια ενός μεμονωμένου δείγματος αισθητήρα δεν
επηρεάζει σημαντικά τη συνολική ανάλυση, επομένως η πολιτική
"\textbf{at-least-once}" αποτελεί μια ασφαλή και επαρκή επιλογή. Οι Kreps et
al. \cite{kafkaoriginal} προσθέτουν, πως το Kafka σχεδιάστηκε εξαρχής με
γνώμονα το υψηλό throughput, αποφεύγοντας περίπλοκους μηχανισμούς όπως το
two-phase commit και υιοθετώντας πιο αποδοτικές λύσεις για περιπτώσεις όπου η
απώλεια μηνυμάτων είναι αποδεκτή.

Τα δεδομένα που εισάγονται στα topics του \textit{Kafka} επεξεργάζονται και
καταλήγουν είτε σε επόμενα topics για μετέπειτα ανάλυση, είτε απευθείας σε
αποθηκευτικά συστήματα. Ο τρόπος με τον οποίο πραγματοποιείται η εν λόγω
επεξεργασία οφείλει να είναι, εν γένει, σε πραγματικό ή σχεδόν πραγματικό
χρόνο, καθώς τα δεδομένα παράγονται και εισάγονται στο σύστημα σε συνθήκες
online ροής. Επομένως, εργαλεία όπως το \textit{Apache Flink}, τα οποία
παρέχουν native υποστήριξη για stream processing με χαμηλή καθυστέρηση,
καθίστανται ιδανικά για την υλοποίηση του ενδιάμεσου επιπέδου επεξεργασίας.

\section{Flink}

Το \textit{Flink} επιτρέπει τον ορισμό παραθύρων (windows) με βάση τον χρόνο
γεγονότος (event time), την υλοποίηση πολύπλοκων λειτουργιών (όπως filtering,
aggregation, enrichment), καθώς και την ενσωμάτωση με messaging και
αποθηκευτικά συστήματα, μεταξώ των οποίων και το \textit{Apache Kafka} και το
\textit{Apache Cassandra}. Επιπλέον, μέσω του μηχανισμού state management που
διαθέτει, διασφαλίζεται η αξιοπιστία της επεξεργασίας, ακόμη και σε περιπτώσεις
προσωρινών αποτυχιών. Η ενσωμάτωση του σε αρχιτεκτονικές τύπου CPS επιτρέπει τη
δημιουργία πραγματικά αντιδραστικών συστημάτων, τα οποία μπορούν να λαμβάνουν
αποφάσεις βάσει εξελισσόμενων δεδομένων, χωρίς να απαιτείται off-line
επεξεργασία ή χρονική υστέρηση.

\section{Cassandra}

Η αποθήκευση μεγάλου όγκου δεδομένων σε κατανεμημένα συστήματα βασίζεται -
κυρίως - σε βάσεις δεδομένων τύπου \textit{wide-column}, με το \textit{Apache
	Cassandra} να αποτελεί ένα από τα πιο διαδεδομένα και ώριμα συστήματα σε αυτόν
τον χώρο. Το Cassandra ακολουθεί το μοντέλο \textit{eventual consistency},
υποστηρίζει κατανεμημένη αποθήκευση με replication και partitioning, και έχει
σχεδιαστεί για \textit{write-heavy} εφαρμογές \cite{cassandrawp}. Σε
περιβάλλοντα IoT, όπου οι συσκευές παράγουν συνεχώς δεδομένα τηλεμετρίας (π.χ.
θερμοκρασία, τάση, ρεύμα) με υψηλή συχνότητα, η Cassandra μπορεί να
λειτουργήσει ως backend αποθήκευσης για ροές δεδομένων σχεδόν σε πραγματικό
χρόνο, διασφαλίζοντας υψηλή διαθεσιμότητα και συνεχή εγγραφή με χαμηλό latency.
Η προσέγγιση αυτή έχει εφαρμοστεί επιτυχώς σε σενάρια όπως η παρακολούθηση
φωτοβολταϊκών μονάδων μέσω Raspberry Pi \cite{iotcassandra}, όπου το σύστημα
συλλέγει και αποθηκεύει μετρήσεις αισθητήρων κάθε 15 λεπτά για περαιτέρω
ανάλυση και βελτιστοποίηση απόδοσης. Αντίστοιχες αρχιτεκτονικές υποδεικνύουν τη
σημασία της επιλογής αποθηκευτικού συστήματος που να ανταποκρίνεται τόσο σε
επιχειρησιακές ανάγκες χαμηλής καθυστέρησης όσο και σε απαιτήσεις αξιοπιστίας
και επεκτασιμότητας.

Στη συγκεκριμένη περίπτωση, η υψηλή απόδοση (high performance) της
\textit{Cassandra} σε συνδυασμό με την εύκολη και ελαστική επεκτασιμότητα
(elastic scalability), καθιστά το σύστημα ιδανικό για εφαρμογές πραγματικού
χρόνου με έντονη ροή δεδομένων. Η δυνατότητα προσθήκης ή αφαίρεσης κόμβων
(nodes) χωρίς διακοπή λειτουργίας επιτρέπει την ομαλή προσαρμογή στις
αυξομειώσεις φορτίου, επιτρέπωντας το self-healing, διατηρώντας παράλληλα
σταθερή τη χρονική απόκριση. Καθώς το σύστημα απαιτεί ταχεία επεξεργασία, άμεση
καταχώρηση και αξιόπιστη αποθήκευση δεδομένων αισθητήρων πεδίου, η επιλογή μιας
αρχιτεκτονικής βασισμένης στην \textit{Cassandra} εξασφαλίζει υψηλή
διαθεσιμότητα και ανθεκτικότητα σε αποτυχία. Το λειτουργικό όφελος που
προκύπτει από αυτήν τη σχεδίαση δεν είναι απλώς επιθυμητό αλλά κρίσιμης
σημασίας, ιδιαίτερα σε περιβάλλοντα με απαιτήσεις χαμηλού latency, συνεχούς
διαθεσιμότητας και γραμμικής επεκτασιμότητας.

\section{Caching}

Τελικός στόχος του συστήματος Nostradamus είναι η αξιοποίηση των αποθηκευμένων
και επεξεργασμένων δεδομένων σε εφαρμογές διεπαφής, ώστε οι χρήστες να μπορούν
να λαμβάνουν οπτικοποιημένη και άμεσα αξιοποιήσιμη πληροφόρηση σχετικά με την
καλλιέργειά τους. Οι κόμβοι της βάσης δεδομένων διαδραματίζουν κρίσιμο ρόλο
στην τροφοδότηση των εφαρμογών με δεδομένα σε πραγματικό χρόνο. Η ανάγκη για
άμεση προσπέλαση σε δεδομένα, ιδιαίτερα σε περιβάλλοντα όπου η εισροή
πληροφορίας είναι συνεχής και εν δυνάμει μαζική, καθιστά απαραίτητη τη χρήση
μηχανισμών προσωρινής αποθήκευσης (in-memory caching).

Στο πλαίσιο αυτό, τεχνολογίες όπως το \textit{Memcached} \cite{memcachedfb} και
το \textit{Redis} \cite{redisia} προσφέρουν σημαντικά πλεονεκτήματα, μέσω της
υλοποίησης γρήγορων και αποδοτικών μηχανισμών αποθήκευσης ζευγών κλειδιού-τιμής
(key-value pairs). Οι εν λόγω λύσεις συμβάλλουν ουσιαστικά στη μείωση του
χρόνου και του κόστους προσπέλασης σε δεδομένα που διαφορετικά θα απαιτούσαν
αναζήτηση στη βάση. Η βασική τους διαφοροποίηση εντοπίζεται κυρίως στο
αρχιτεκτονικό τους μοντέλο και στον τρόπο διαχείρισης της ταυτόχρονης
εκτέλεσης, με το \textit{Memcached} να βασίζεται σε multithreaded επεξεργασία,
ενώ το \textit{Redis} αξιοποιεί μοντέλο event loop.

\section{Kubernetes}

Η ανάγκη για παρατηρησιμότητα, αποδοτική διαχείριση προσωρινών δεδομένων και
σταθερή ροή μηνυμάτων σε περιβάλλοντα με κατανεμημένη υπολογιστική λογική,
καθιστά την πλατφόρμα \textit{Kubernetes}, συχνά, απαραίτητο δομικό στοιχείο. Η
δυνατότητα αυτόματης επανεκκίνησης αποτυχημένω πόρων (self-healing), η
υποστήριξη οριζόντιας κλιμάκωσης (horizontal pod autoscaling) και η
ενσωματωμένη παρακολούθηση της κατάστασης των υπηρεσιών (liveness/readiness
probes), προσδίδουν λειτουργική ανθεκτικότητα και διατηρούν τη διαθεσιμότητα
του συστήματος σε υψηλά επίπεδα, ακόμη και σε συνθήκες έντονου φορτίου ή υλικών
αποτυχιών.

Πέραν της αυτοματοποιημένης διαχείρισης πόρων, το \textit{Kubernetes} παρέχει
μια ενιαία πλατφόρμα παρατηρησιμότητας. Μέσω της ενσωμάτωσης εργαλείων όπως το
\textit{Prometheus} για συλλογή μετρικών, το \textit{Grafana} για οπτικοποίηση
και το \textit{AlertManager} για την αποστολή ειδοποιήσεων, \cite{inframon} -
κοινώς τη \textbf{lingua franca} του \textit{observability} - ενισχύεται η
κατανόηση της δυναμικής συμπεριφοράς του συστήματος σε πραγματικό χρόνο. Η
εγγενής υποστήριξη μηχανισμών για service discovery, load balancing και
declarative configuration, επιτρέπει την ευέλικτη ανάπτυξη και τον συντονισμό
μικροϋπηρεσιών, διευκολύνοντας την επίτευξη στόχων υψηλής διαθεσιμότητας και
επεκτασιμότητας σε cloud-native περιβάλλοντα. Συνεπώς, υιοθετώντας αντίστοιχα
πρότυπα, μπορούν να προληφθούν ανεπιθύμητες καταστάσεις και να διασφαλιστεί η
διατήρηση της ομαλής λειτουργίας ακόμα και υπο συνθήκες υψηλής πολυπλοκότητας.
