\chapter{Πειράματα}
\label{chap:experiments}

Η παρούσα ενότητα εξετάζει τη συμπεριφορά των μηχανισμών προσωρινής αποθήκευσης
που υποστηρίζουν το endpoint \textit{/aggregate}, το οποίο αποτελεί βασικό
συστατικό της πλατφόρμας \textit{Nostradamus}. Η υπηρεσία αυτή υλοποιεί
παραθυρικές συναθροίσεις σε χρονοσειρές αισθητήρων και λειτουργεί ως μεσαίο
επίπεδο μεταξύ της εφαρμογής και της μόνιμης βάσης δεδομένων. Η καθυστέρηση
ανάγνωσης και εγγραφής στο cache layer μεταφέρεται άμεσα στον χρόνο απόκρισης
του API, επομένως η απόδοση του εκάστοτε υποστρώματος αποτελεί πρωτεύοντα
παράγοντα για τη συνολική σταθερότητα της πλατφόρμας.

\section{Διατύπωση στόχου}

Στόχος της πειραματικής διαδικασίας είναι η αποτίμηση της απόδοσης και της
σταθερότητας δύο πλήρως εναλλάξιμων cache drivers (\textit{Valkey},
\textit{Memcached}) όταν χρησιμοποιούνται από το \textit{/aggregate} υπό
συνθήκες συνεχούς φόρτου. Η αξιολόγηση επικεντρώνεται σε τρεις παραμέτρους:

\begin{enumerate}
	\item \textbf{Απόδοση ανάγνωσης και εγγραφής}. Εξετάζονται συμμετρικά τα
		μονοπάτια \textit{cache hit} και \textit{cache miss}, καθώς
		αμφότερα συνεισφέρουν στον τελικό χρόνο απόκρισης.
	\item \textbf{Σταθερότητα σε υψηλά percentiles}. Για συστήματα
		πραγματικού χρόνου, η μέση τιμή είναι ανεπαρκής δείκτης· τα
		υψηλά percentiles (P95) είναι εκεί όπου εκδηλώνονται οι χρονικές
		αποκλίσεις που επηρεάζουν ουσιαστικά την εμπειρία χρήστη.
	\item \textbf{Συμπεριφορά σε περιβάλλοντα συχνής ανανέωσης}. Η
		χαμηλή χρονική διάρκεια ζωής (TTL) σε χρονοσειρές αισθητήρων
		προκαλεί συχνές invalidations, επομένως ο driver πρέπει να
		ανταποκρίνεται σωστά σε workloads που χαρακτηρίζονται από
		επαναλαμβανόμενη εισαγωγή νέων τιμών.
\end{enumerate}

Οι παραπάνω στόχοι αποτυπώνουν τις πραγματικές απαιτήσεις ενός αγροτικού
deployment, όπου οι συσκευές παράγουν συνεχή ροή δεδομένων και το API πρέπει να
διατηρεί σταθερή συμπεριφορά ακόμα και υπό μεταβαλλόμενο φόρτο.

\section{Μεθοδολογία}

Για την απομόνωση των χαρακτηριστικών κάθε υποστρώματος δημιουργήθηκαν δύο
ανεξάρτητες instantiations της πλατφόρμας: μία που χρησιμοποιεί \textit{Valkey}
και μία που χρησιμοποιεί \textit{Memcached}. Και οι δύο παραμετροποιήθηκαν μέσω
του κοινού cache interface, ώστε η αλλαγή driver να μην επηρεάζει την
εφαρμοστική λογική του \textit{/aggregate}.

Ο φόρτος παράχθηκε μέσω synthetic load generator, ο οποίος μιμείται το προφίλ
αιτημάτων πραγματικού deployment: επαναλαμβανόμενα queries σε συγκεκριμένα
παράθυρα (15 λεπτών έως 1 ώρας) για διαφορετικούς τύπους αισθητήρων, με σταθερό
ρυθμό αποστολής. Η συλλογή των μετρήσεων πραγματοποιήθηκε μέσω
\textit{Prometheus}, ενώ η οπτικοποίηση έγινε στο \textit{Grafana} με έμφαση
στο P95 latency, στις write καθυστερήσεις και στο ποσοστό invalidations.

Η παραμετροποίηση των drivers παρέμεινε στην προεπιλεγμένη μορφή τους, εκτός
από το TTL των εγγραφών που ορίστηκε στα 120~δευτερόλεπτα, ώστε να προσεγγίζει
τον λειτουργικό ορίζοντα των μετρήσεων πεδίου.

\section{Αποτελέσματα πειράματος}

Τα αποτελέσματα καταδεικνύουν σαφή διαφοροποίηση ως προς τη συμπεριφορά των δύο
υποστρωμάτων. Το \textit{Memcached} εμφανίζει χαμηλότερη καθυστέρηση σε
αναγνώσεις και εγγραφές, περίπου $1.7\times$ ταχύτερο από το \textit{Valkey}
στο P95. Η επίδοση αυτή συνδέεται με τον λιτό χαρακτήρα του πρωτοκόλλου του
(Memcached binary protocol), το οποίο αποφεύγει πολυπλοκότητα και επιτρέπει
υψηλό throughput με περιορισμένο αποτύπωμα πόρων.

Αντίθετα, το \textit{Valkey} υπερέχει ως προς τη χρονική σταθερότητα: οι
καμπύλες latency παραμένουν σχεδόν σταθερές ακόμη και όταν αυξάνεται το
concurrency, χωρίς την εμφάνιση αιχμών. Η συμπεριφορά αυτή είναι σημαντική για
workloads που απαιτούν προβλεψιμότητα ή κάνουν χρήση λειτουργιών που
απουσιάζουν από το \textit{Memcached} (π.χ. atomic blocks, Lua scripts,
σύνθετες δομές δεδομένων).

Ως εκ τούτου, τα δύο συστήματα δεν αποτελούν εναλλακτικές που διαφέρουν απλώς
ποσοτικά, αλλά ποιοτικά: το \textit{Memcached} επιτυγχάνει υψηλότερη
ακατέργαστη ταχύτητα, ενώ το \textit{Valkey} προσφέρει συνεπή συμπεριφορά ακόμη
και σε περιόδους υψηλής συμφόρησης και διαθέτει επεκτασιμότητα που το καθιστά
καταλληλότερο για μελλοντική ενσωμάτωση πιο σύνθετων λειτουργιών της
πλατφόρμας.

Με δεδομένη τη modular αρχιτεκτονική, η επιλογή driver μπορεί να καθοδηγείται
από τις επιχειρησιακές ανάγκες κάθε deployment, χωρίς καμία αλλαγή στον κώδικα
της εφαρμογής.

\section{Περαιτέρω επεκτάσεις}

Η τρέχουσα πειραματική διαδικασία αποτελεί θεμέλιο για μελλοντική ανάλυση. Ως
επεκτάσεις σχεδιάζονται:

\begin{itemize}
	\item διερεύνηση της επίδρασης των invalidations σε διαφορετικούς ρυθμούς εισαγωγής
		νέων μετρήσεων,
	\item αποτύπωση κατανάλωσης μνήμης και CPU ανά driver,
	\item μελέτη συμπεριφοράς σε συνθήκες οριζόντιας κλίμακωσης,
\end{itemize}

Οι παραπάνω επεκτάσεις αποτελούν απαραίτητη συνέχεια για την πλήρη χαρτογράφηση
της συμπεριφοράς των υποστρωμάτων και για την τεκμηρίωση των επιχειρησιακών
αποφάσεων που θα καθορίσουν τη μελλοντική μορφή της πλατφόρμας.
