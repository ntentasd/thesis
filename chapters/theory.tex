\chapter{Θεωρία}

Η ανάγκη για αποδοτική ροή δεδομένων μεταξύ μικροϋπηρεσιών οδηγεί συνήθως στην υιοθέτηση τεχνολογιών όπως το \textit{Apache Kafka}, ένα - πλέον - de facto πρότυπο για την υλοποίηση ανθεκτικών και επεκτάσιμων messaging υποδομών, χάρη στη δυνατότητά του να αποθηκεύει τα μηνύματα σε μορφή καταγραφής (log-based) \cite{kafkabdd}. Η αρχιτεκτονική του Kafka ενδείκνυται ιδιαίτερα για περιβάλλοντα που απαιτούν real ή near-real time επεξεργασία γεγονότων, καθώς παρέχει υψηλή διαθεσιμότητα, εγγυημένη διανομή μηνυμάτων και δυνατότητα οριζόντιας κλιμάκωσης. Συγκριτικές μελέτες \cite{rtkafka} επιβεβαιώνουν ότι το Kafka παρουσιάζει σημαντικό πλεονέκτημα σε όρους throughput και fault tolerance σε σχέση με άλλες προσεγγίσεις, γεγονός που το καθιστά κατάλληλο για απαιτητικές IoT εφαρμογές με αυξημένο όγκο και ταχύτητα δεδομένων. Στο πλαίσιο αυτό, το Kafka συνιστά θεμελιώδες δομικό στοιχείο για αρχιτεκτονικές που στοχεύουν στην ημι-πραγματική επεξεργασία δεδομένων, όπως το σύστημα \textit{Nostradamus}.

Η αποθήκευση μεγάλου όγκου δεδομένων σε κατανεμημένα συστήματα βασίζεται - κυρίως - σε βάσεις δεδομένων τύπου \textit{wide-column}, με το \textit{Apache Cassandra} να αποτελεί ένα από τα πιο διαδεδομένα και ώριμα συστήματα σε αυτόν τον χώρο. Το Cassandra ακολουθεί το μοντέλο \textit{eventual consistency}, υποστηρίζει κατανεμημένη αποθήκευση με replication και partitioning, και έχει σχεδιαστεί για \textit{write-heavy} εφαρμογές \cite{cassandrawp}. Σε περιβάλλοντα IoT, όπου οι συσκευές παράγουν συνεχώς δεδομένα τηλεμετρίας (π.χ. θερμοκρασία, τάση, ρεύμα) με υψηλή συχνότητα, η Cassandra μπορεί να λειτουργήσει ως backend αποθήκευσης για ροές δεδομένων σχεδόν σε πραγματικό χρόνο, διασφαλίζοντας υψηλή διαθεσιμότητα και συνεχή εγγραφή με χαμηλό latency. Η προσέγγιση αυτή έχει εφαρμοστεί επιτυχώς σε σενάρια όπως η παρακολούθηση φωτοβολταϊκών μονάδων μέσω Raspberry Pi \cite{iotcassandra}, όπου το σύστημα συλλέγει και αποθηκεύει μετρήσεις αισθητήρων κάθε 15 λεπτά για περαιτέρω ανάλυση και βελτιστοποίηση απόδοσης. Αντίστοιχες αρχιτεκτονικές υποδεικνύουν τη σημασία της επιλογής αποθηκευτικού συστήματος που να ανταποκρίνεται τόσο σε επιχειρησιακές ανάγκες χαμηλής καθυστέρησης όσο και σε απαιτήσεις αξιοπιστίας και επεκτασιμότητας.

Σε ότι αφορά την ανάγνωση των αποθηκευμένων δεδομένων αυτού του όγκου, αλλά και δεδομένης της συνεχόμενης άφιξης τους, τα προσωρινά δεδομένα σε μνήμη (\textit{in-memory caching}) αποτελούν κρίσιμο μηχανισμό βελτιστοποίησης της απόδοσης. Δυο από τις πιο διαδεδομένες τεχνολογίες στον τομέα αυτό είναι τα συστήματα \textit{Memcached} \cite{memcachedfb} και \textit{Redis} \cite{redisia}, τα οποία υλοποιούν μηχανισμούς προσωρινής αποθήκευσης ζευγών κλειδιού-τιμής, με στόχο τη μείωση του χρόνου και του κόστους προσπέλασης σε βάσεις δεδομένων. Η βασική διαφορά τους έγκειται κυρίως στην αρχιτεκτονική του ολικού συστήματος, καθώς και στον τρόπο διαχείρισης της ταυτόχρονης εκτέλεσης - με χρήση \textit{multithreading} (Memcached) ή \textit{event loop} (Redis).

Η ανάγκη για παρατηρησιμότητα, αποδοτική διαχείριση προσωρινών δεδομένων και σταθερή ροή μηνυμάτων σε περιβάλλοντα με κατανεμημένη υπολογιστική λογική, καθιστά την πλατφόρμα \textit{Kubernetes} - συχνά -  απαραίτητο δομικό στοιχείο. Η χρήση της επιτρέπει την υλοποίηση μηχανισμών αυτόματης αποκατάστασης, \textit{autoscaling}, και επιτήρησης σε επίπεδο υπηρεσίας. Σε συνδυασμό με συστήματα όπως το \textit{Prometheus} (και τα συνεργατικά του υποσυστήματα \textit{Grafana} και \textit{AlertManager}) \cite{inframon} - κοινώς τη lingua franca του \textit{observability} - συμβάλλει στον σχεδιασμό και στην υλοποίηση μεγάλων υποδομών, πληρώντας τα κριτήρια μιας παραγωγικής (production-grade) αρχιτεκτονικής.
