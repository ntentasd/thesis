\chapter{Θεωρητικό Υπόβαθρο}

Η εξέλιξη των Κυβερνο-Φυσικών Συστημάτων (CPS) και των εφαρμογών IoT βασίζεται σε θεμελιώδεις αρχιτεκτονικές αρχές που συνδυάζουν την αποδοτική διαχείριση της φυσικής πληροφορίας με τις δυνατότητες της σύγχρονης υπολογιστικής τεχνολογίας. Σε αυτό το κεφάλαιο παρουσιάζονται τα βασικά αρχιτεκτονικά μοντέλα, οι μέθοδοι διασύνδεσης υποσυστημάτων και τα πρότυπα διαχείρισης δεδομένων που επικρατούν στα σύγχρονα CPS.

\section{Αρχιτεκτονικές CPS και Κατανεμημένα Συστήματα}

Τα CPS δομούνται τυπικά σε τρία επίπεδα: το \textbf{φυσικό επίπεδο} (sensing/actuation), το \textbf{επίπεδο δικτύου} (networking/communication) και το \textbf{κυβερνητικό ή υπολογιστικό επίπεδο} (cyber/computation). Το φυσικό επίπεδο περιλαμβάνει αισθητήρες και ενεργοποιητές, το δίκτυο μεταφέρει τα δεδομένα, ενώ το κυβερνητικό επίπεδο διαχειρίζεται την επεξεργασία, την ανάλυση και τη λήψη αποφάσεων.

Η σύγχρονη σχεδίαση CPS ακολουθεί τα παρακάτω αρχιτεκτονικά μοτίβα:

\begin{itemize}
	\item \textbf{Layered (στρωματοποιημένη) αρχιτεκτονική}: Διαχωρισμός φυσικών, δικτυακών και υπολογιστικών λειτουργιών, διευκολύνοντας τη διαλειτουργικότητα και τη συντήρηση.
	\item \textbf{Κατανεμημένη επεξεργασία}: Τα δεδομένα δεν συγκεντρώνονται υποχρεωτικά σε έναν κεντρικό κόμβο, αλλά επεξεργάζονται τοπικά (\textit{edge/fog computing}) ή υβριδικά, βελτιώνοντας την καθυστέρηση και την ανθεκτικότητα.
	\item \textbf{Event-driven και streaming αρχιτεκτονική}: Αντί της παραδοσιακής batch επεξεργασίας, τα δεδομένα ρέουν ως ακολουθίες γεγονότων (streams), επιτρέποντας την άμεση αντίδραση σε αλλαγές του περιβάλλοντος.
	\item \textbf{Μικροϋπηρεσίες (Microservices)}: Ανάπτυξη του λογισμικού σε μικρές, αυτόνομες υπηρεσίες με σαφή όρια ευθύνης και ανεξάρτητο κύκλο ζωής.
	\item \textbf{Loose coupling \& αναγνωσιμότητα}: Η επικοινωνία μεταξύ των υποσυστημάτων βασίζεται σε ασύγχρονα μηνύματα (π.χ. μέσω pub/sub patterns), διατηρώντας την αυτονομία και διευκολύνοντας την κλιμάκωση.
\end{itemize}

\section{Streams, Event-Driven και Batch vs Real-Time Processing}

Η \textit{ροή δεδομένων} (stream) ορίζεται ως η συνεχής αλληλουχία δεδομένων που παράγονται από αισθητήρες ή άλλες πηγές και διακινούνται ασύγχρονα εντός του συστήματος. Η \textbf{event-driven αρχιτεκτονική} επιτρέπει τη λήψη αποφάσεων ή την ενεργοποίηση ενεργειών αμέσως με την εμφάνιση ενός γεγονότος, κάτι που είναι κρίσιμο για εφαρμογές με απαιτήσεις απόκρισης σε πραγματικό χρόνο, όπως η γεωργία ακριβείας, το predictive maintenance, ή η αυτονομία ρομποτικών συστημάτων.

Σε αντίθεση, τα παραδοσιακά \textit{batch systems} επεξεργάζονται δεδομένα συγκεντρωτικά και περιοδικά, οδηγώντας σε υστέρηση που δεν είναι αποδεκτή για CPS με κρίσιμες λειτουργίες. Η υιοθέτηση stream processing (με αρχιτεκτονικές Lambda/Kappa) προσφέρει καλύτερη προσαρμοστικότητα και αυτονομία, ενώ μειώνει την καθυστέρηση λήψης αποφάσεων.

\section{Αρχιτεκτονικές Streams: Lambda και Kappa}

Οι δύο επικρατέστερες αρχιτεκτονικές για επεξεργασία ροών είναι:

\begin{itemize}
	\item \textbf{Lambda}: Συνδυάζει speed layer για άμεση επεξεργασία (με πιθανό μικρότερο accuracy) και batch layer για πλήρη ανάλυση με υψηλότερη ακρίβεια, υποστηρίζοντας τόσο real-time όσο και offline analytics.
	\item \textbf{Kappa}: Ενιαία προσέγγιση, όπου όλα τα δεδομένα αντιμετωπίζονται ως streams, απλοποιώντας τη διαχείριση και επανεπεξεργασία γεγονότων χωρίς ανάγκη ξεχωριστού batch υποσυστήματος.
\end{itemize}

Η επιλογή μοντέλου εξαρτάται από το αν προτεραιότητα δίνεται στην αμεσότητα ή στην αναλυτική επεξεργασία και από τη συνολική πολυπλοκότητα συντήρησης. Σε πρακτικό επίπεδο, υλοποιήσεις όπως το \textit{Apache Kafka} (messaging/stream log), \textit{Apache Flink} και \textit{Apache Spark} (stream processing) έχουν επικρατήσει ως τεχνολογικά standards, λόγω της ευελιξίας, αξιοπιστίας και δυνατότητας διασύνδεσης με άλλα υποσυστήματα.

\section{Διασύνδεση Υποσυστημάτων}

Για την αλληλεπίδραση πολλαπλών, ετερογενών υποσυστημάτων, υιοθετείται το \textbf{publish/subscribe} μοτίβο. Αυτό επιτρέπει την ασύγχρονη και χαλαρά συνδεδεμένη (\textit{loosely coupled}) επικοινωνία μεταξύ παραγωγών (\textit{publishers}) και καταναλωτών (\textit{subscribers}), διασφαλίζοντας την αυτονομία, την επεκτασιμότητα και τη δυνατότητα αλλαγών στην τοπολογία του συστήματος χωρίς κεντρικό συντονισμό.

Η ποιότητα υπηρεσίας (\textit{Quality of Service, QoS}) αποκτά ιδιαίτερη σημασία, καθώς καθορίζει το επίπεδο αξιοπιστίας στη διανομή μηνυμάτων, ειδικά σε περιβάλλοντα με ασταθή συνδεσιμότητα (π.χ. αγροτικά δίκτυα, βιομηχανικά IoT). Για ποιοτικά περιορισμένες συσκευές και low-power δίκτυα, πρωτόκολλα όπως το \textit{MQTT} προσφέρουν ελαφριά, αξιόπιστη και ασφαλή ανταλλαγή μηνυμάτων, ενισχύοντας την ανθεκτικότητα του συστήματος.

\section{Αποθήκευση, Consistency Models και Caching}

Η αποθήκευση δεδομένων σε CPS συνιστά αρχιτεκτονική πρόκληση λόγω του όγκου, της ταχύτητας παραγωγής και της ανάγκης για real-time προσπέλαση. Κατανεμημένα συστήματα βάσεων (NoSQL, wide-column stores) με replication και partitioning διασφαλίζουν διαθεσιμότητα και fault tolerance.

Η επιλογή consistency model (strong, eventual, causal) εξαρτάται από τις απαιτήσεις της εφαρμογής για ακρίβεια vs ταχύτητα και διαθεσιμότητα. Σε συστήματα που μπορούν να αντέξουν μικρές απώλειες ή προσωρινή ασυνέπεια, το eventual consistency αποτελεί πρακτικό συμβιβασμό.

Επισημαίνεται επίσης, ότι η χρήση \textit{in-memory caching} (π.χ. με Redis/Memcached) μειώνει δραστικά το latency για δεδομένα που προσπελαύνονται συχνά ή απαιτούν άμεση διαθεσιμότητα, με trade-offs σε αντοχή σε αποτυχίες (durability).

\section{Διαχείριση Υποδομής και Cloud-Native Patterns}

Η ανάπτυξη CPS σε cloud ή υβριδικά περιβάλλοντα απαιτεί αυτοματοποιημένη διαχείριση υποδομής και υπηρεσιών. Εδώ κυριαρχεί η χρήση πλατφόρμων όπως το Kubernetes, που υλοποιούν:

\begin{itemize}
	\item \textbf{Δηλωτική διαχείριση (declarative infrastructure)}: Η επιθυμητή κατάσταση του συστήματος ορίζεται μέσω configuration, και ο orchestrator διασφαλίζει ότι αυτή τηρείται.
	\item \textbf{Αυτόματη κλιμάκωση (auto-scaling), αυτοΐαση (self-healing)}: Προσθήκη/αφαίρεση pods, επανεκκινήσεις σε περίπτωση αποτυχιών, health checks.
	\item \textbf{Παρατηρησιμότητα (observability)}: Συλλογή μετρικών, logs, tracing και alerting, με ενδεικτικά εργαλεία Prometheus, Grafana, ELK/EFK stack.
	\item \textbf{Διαμοιρασμός πόρων (virtualization)}: Ευνοϊκή μεθοδολογία για την απομόνωση εφαρμογών και την καλύτερη αξιοποίηση του διαθέσιμου hardware μέσω εικονικών μηχανών ή containers.
	\item \textbf{Αυτοματοποιημένος έλεγχος με Kubernetes Operators}: Επέκταση του control plane του Kubernetes, με αρχές αυτοματοποιημένων συστημάτων ελέγχου για τη διαχείριση σύνθετων, κατανεμημένων εφαρμογών.
	\item \textbf{Service discovery, load balancing και secrets management}: Απαραίτητα για τη διασύνδεση μικροϋπηρεσιών και τη διατήρηση ασφάλειας και διαθεσιμότητας.
\end{itemize}

Έτσι, η υποδομή μεταμορφώνεται σε έναν ζωντανό οργανισμό που προσαρμόζεται διαρκώς στις απαιτήσεις του συστήματος και του περιβάλλοντος, προσφέροντας ευελιξία και ανθεκτικότητα χωρίς ανθρώπινη παρέμβαση. Με αυτόν τον τρόπο, περιορίζονται τα λάθη του ανθρώπινου παράγοντα, καθώς κρίσιμες λειτουργίες αυτοματοποιούνται και αναλαμβάνονται από το ίδιο το σύστημα.

% \section{Συμπεράσματα Αρχιτεκτονικής}
%
% Η επιτυχία ενός σύγχρονου CPS/IoT συστήματος εξαρτάται από την υιοθέτηση δοκιμασμένων αρχιτεκτονικών προτύπων, τη σαφή οριοθέτηση των στρωμάτων (sensing, networking, processing, storage, observability), και τη χρήση τεχνολογιών που υποστηρίζουν real-time streaming, fault tolerance, scalability, automation και ασφάλεια. Οι επιλογές σε επίπεδο εργαλείων (Kafka, Flink, Cassandra, Redis, Kubernetes, Prometheus, κτλ.) υλοποιούν στην πράξη τα θεμελιώδη design patterns που περιγράφηκαν σε αυτό το κεφάλαιο, αποτελώντας τη “γέφυρα” μεταξύ θεωρίας και εφαρμογής.
