\chapter{Συμπεράσματα}

Η παρούσα διπλωματική εργασία πέτυχε τον κύριο στόχο της, που ήταν η ποσοτική
αξιολόγηση των επιδόσεων συστημάτων γεωργίας ακριβείας υπό ρεαλιστικές συνθήκες
φόρτου, και η διερεύνηση της επίδρασης των δομικών του αποφάσεων (όπως το
\textit{TTL} και το \textit{caching}) στις βασικές μετρικές καθυστέρησης. Τα
συμπεράσματα που προέκυψαν από την ανάλυση των επιμέρους συστατικών της
πλατφόρμας (ScyllaDB, Kafka, EMQX, Arroyo) συνοψίζονται ως εξής:

\section{Αρχιτεκτονική Ανθεκτικότητα και Επεξεργασία Ροών}

Η υιοθέτηση μιας σύγχρονης, \textbf{Cloud-Native αρχιτεκτονικής} αποτέλεσε τον
θεμέλιο λίθο για τη διαχείριση υψηλών ρυθμών εισαγωγής δεδομένων. Η χρήση του
\textit{Kafka} ως κεντρικού διαύλου ενοποίησης επέτρεψε την ανεξάρτητη
οριζόντια κλιμάκωση των παραγωγών και των καταναλωτών, διασφαλίζοντας παράλληλα
μηχανισμούς \textit{back-pressure} για την αποφυγή απώλειας πληροφορίας.

Η αλυσίδα επεξεργασίας ροών \textbf{EMQX $\rightarrow$ Kafka $\rightarrow$
Arroyo} λειτούργησε ως η κύρια ραχοκοκαλιά της πλατφόρμας. Το \textit{Kafka}
διαχειρίστηκε με επιτυχία την \textbf{ανεξάρτητη επεκτασιμότητα}
(\textit{horizontal scalability}) των \textit{producers} (EMQX) και των
\textit{consumers} (Arroyo/ScyllaDB), διασφαλίζοντας παράλληλα μηχανισμούς
\textbf{back-pressure} και την \textbf{μη-απώλεια} δεδομένων. Επιπλέον, η χρήση
του \textit{Arroyo} επιβεβαίωσε τη δυνατότητα για \textbf{stateful} επεξεργασία
με \textbf{\textit{exactly-once} semantics}, γεγονός κρίσιμο για την αξιοπιστία
των εξαγόμενων χρονοσειρών σε εφαρμογές γεωργίας ακριβείας.

\section{Βελτιστοποίηση Απόδοσης και Διαχείριση Cache}

Τα βασικά υποσυστήματα επέδειξαν εξαιρετικές επιδόσεις, με τη \textit{ScyllaDB}
να διατηρεί το \textbf{P95 latency} σε τιμές \textbf{<20ms} ακόμη και υπό φόρτο
\textbf{350 RPS}. Η έρευνα ανέδειξε τη σημασία του επιπέδου προσωρινής
αποθήκευσης μέσω δύο αξόνων:

\begin{itemize}
	\item \textbf{Στρατηγική Επιλογή Cache Drivers:} Η σύγκριση μεταξύ
		\textit{Valkey} και \textit{Memcached} κατέδειξε ένα σαφές
		\textit{trade-off}. Ενώ το \textit{Memcached} υπερέχει σε
		απόλυτο \textit{throughput}, η ανθεκτικότητα του
		\textit{Valkey} στις ακραίες τιμές καθυστέρησης (\textit{P99
		tail latency}) τον καθιστά καταλληλότερο για συστήματα
		πραγματικού χρόνου που απαιτούν προβλεψιμότητα.
	\item \textbf{Data Granularity και TTL:} Διαπιστώθηκε ότι η ενεργός
		διαχείριση της επικαιρότητας μέσω \textit{TTL} και
		\textit{invalidations} είναι βιώσιμη χωρίς σημαντική
		επιβάρυνση. Αντίθετα, η υποβέλτιστη πυκνότητα δεδομένων από
		τους αισθητήρες μπορεί να αυξήσει την καθυστέρηση εγγραφής στην
		\textit{cache} έως και 56\%, αναδεικνύοντας την ανάγκη για
		ολιστική παραμετροποίηση από το πεδίο έως το σύννεφο.
\end{itemize}

\section{Ολοκληρωμένο Stack Παρατηρησιμότητας}

Η υλοποίηση του πλαισίου \textbf{monitoring} με \textit{Prometheus} και
\textit{Grafana} επέτρεψε την ποσοτική ανάλυση της απόδοσης, αναδεικνύοντας
φαινόμενα όπως το \textbf{jitter} στην καθυστέρηση εγγραφής λόγω διεργασιών
συντήρησης (GC).

Η στρατηγική επέκταση με το \textbf{Loki} (Logs) και το \textbf{Tempo} με
\textit{OpenTelemetry} (Distributed Tracing) ολοκληρώνει ένα πλήρες
\textbf{Observability Stack}. Η τριπλή αυτή προσέγγιση επιτυγχάνει:
\begin{itemize} \item \textbf{Tracing:} Τον εντοπισμό καθυστερήσεων σε επίπεδο
υπηρεσίας, επιτρέποντας την παρακολούθηση ενός αιτήματος σε όλη τη διαδρομή
του. \item \textbf{Root Cause Analysis (RCA):} Την άμεση συσχέτιση των
\textit{logs} με ανωμαλίες στα \textit{metrics} και τα \textit{traces},
επιταχύνοντας τον χρόνο εντοπισμού της ρίζας των προβλημάτων. \end{itemize} Με
αυτόν τον τρόπο, η πλατφόρμα αποκτά την πλήρη διαγνωστική ικανότητα που
απαιτείται για τη λήψη αυτοματοποιημένων αποφάσεων και την υποστήριξη
\textit{self-healing} μηχανισμών.

\section{Μελλοντική Εργασία και Ερευνητικές Προεκτάσεις}

Η ολοκλήρωση της παρούσας εργασίας θέτει τις βάσεις για μια σειρά από
στρατηγικές επεκτάσεις, οι οποίες στοχεύουν στη μετάβαση αντίστοιχων συστημάτων
από ισχυρά αρχιτεκτονικά πρότυπα σε πλήρως αυτόνομα οικοσυστήματα βιομηχανικού
επιπέδου. Οι προτεινόμενες κατευθύνσεις συνοψίζονται στους εξής άξονες:

\subsection{Αρχιτεκτονική Αυτονομία και Δυναμικό Self-Healing}

Η πλήρης αξιοποίηση του εγκατεστημένου \textit{Observability Stack} (Metrics,
Logs, Traces) επιτρέπει τη μετάβαση από την παθητική επιτήρηση στην ενεργή
αυτοθεραπεία του συστήματος.

\begin{itemize}
	\item \textbf{Προσαρμοστικό Circuit-Breaking:} Προτείνεται η
		ενσωμάτωση μηχανισμών που θα απομονώνουν αυτόματα "αργούς"
		κόμβους ή δυσλειτουργικούς \textit{cache drivers} σε πραγματικό
		χρόνο, βασιζόμενοι σε δυναμικά κατώφλια (thresholds) που
		ορίζονται από τα \textit{traces} του \textit{Tempo}.
	\item \textbf{Auto-scaling βάσει Πίεσης (Back-Pressure):} Η ανάπτυξη
		ενορχηστρωτών που θα κλιμακώνουν οριζόντια τους \textit{Arroyo
		workers} ανάλογα με την καθυστέρηση του καταναλωτή (consumer
		lag) στο \textit{Kafka}, διασφαλίζοντας την τήρηση των
		\textit{SLAs} καθυστέρησης ακόμη και σε απότομες αυξήσεις της
		ροής δεδομένων.
\end{itemize}

\subsection{Ευφυής Διαχείριση Cache και Μηχανική Μάθηση}

Η έρευνα στο επίπεδο της προσωρινής αποθήκευσης μπορεί να επεκταθεί πέρα από
τους στατικούς κανόνες \textit{TTL}.

\begin{itemize}
	\item \textbf{Predictive Pre-caching:} Η χρήση μοντέλων Μηχανικής
		Μάθησης (όπως \textit{LSTMs} ή \textit{Graph Neural Networks})
		για την πρόβλεψη των μοτίβων πρόσβασης των χρηστών μπορεί να
		επιτρέψει τον προ-υπολογισμό κρίσιμων δεδομένων πριν αυτά
		ζητηθούν, μειώνοντας δραστικά το \textit{cold-start latency}.
	\item \textbf{Σημασιολογικό Caching (Semantic Caching):} Ανάπτυξη ενός
		επιπέδου που θα αναγνωρίζει τη σπουδαιότητα των δεδομένων (π.χ.
		προτεραιότητα σε "κρύα" δεδομένα έναντι μετρήσεων ρουτίνας),
		προσαρμόζοντας δυναμικά τις πολιτικές εκκαθάρισης (eviction
		policies).
\end{itemize}

\subsection{Edge-to-Cloud και Γεωγραφική Κλιμάκωση}

Για τη βελτιστοποίηση των πόρων δικτύου και τη μείωση του \textit{latency}, η
επεξεργασία πρέπει να μεταφερθεί εγγύτερα στην πηγή.

\begin{itemize}
	\item \textbf{Αποκεντρωμένο Stream Processing:} Η υλοποίηση μέρους της
		λογικής του \textit{Arroyo} σε \textit{Edge Gateways} εντός του
		αγρού θα επιτρέψει το τοπικό φιλτράρισμα και τη συμπίεση των
		δεδομένων, στέλνοντας στο \textit{Cloud} μόνο τις πληροφορίες
		που απαιτούνται για ιστορική ανάλυση.
	\item \textbf{Multi-region State Consistency:} Διερεύνηση μεθόδων
		συγχρονισμού της κατάστασης (\textit{state consistency}) μεταξύ
		διαφορετικών γεωγραφικών περιοχών, διασφαλίζοντας υψηλή
		διαθεσιμότητα και επιχειρησιακή συνέχεια (\textit{business
		continuity}) ακόμη και σε περιπτώσεις ολοκληρωτικής αστοχίας
		ενός \textit{datacenter}.
	\item \textbf{Multi-cloud Deployment και Vendor Agnosticism:} Ως τελικό
		στάδιο θωράκισης, προτείνεται η διασπορά της υποδομής σε
		ετερογενείς παρόχους νέφους (\textit{Multi-cloud strategy}). Η
		προσέγγιση αυτή εξαλείφει την εξάρτηση από έναν μεμονωμένο
		πάροχο (\textit{vendor lock-in}) και απομονώνει πλήρως το
		σύστημα από γενικευμένες αστοχίες υποδομής σε επίπεδο παρόχου,
		εξασφαλίζοντας το μέγιστο δυνατό επίπεδο ανθεκτικότητας για την
		κρίσιμες υποδομές.
\end{itemize}
