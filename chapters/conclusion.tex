\chapter{Συμπεράσματα}

Η παρούσα διπλωματική εργασία πέτυχε τον κύριο στόχο της, που ήταν η ποσοτική
αξιολόγηση των επιδόσεων του συστήματος \textbf{Nostradamus} υπό ρεαλιστικές
συνθήκες φόρτου, και η διερεύνηση της επίδρασης των δομικών του αποφάσεων (όπως
το \textit{TTL} και το \textit{caching}) στις βασικές μετρικές καθυστέρησης. Τα
συμπεράσματα που προέκυψαν από την ανάλυση των επιμέρους συστατικών της
πλατφόρμας (ScyllaDB, Kafka, EMQX, Arroyo) συνοψίζονται ως εξής:

\begin{enumerate}
	\item \textbf{Cloud-Native Αρχιτεκτονική:} Η υιοθέτηση μιας σύγχρονης,
		\textbf{Cloud-Native αρχιτεκτονικής} με βάση τεχνολογίες όπως
		\textit{Kafka} για την ενοποίηση των δεδομένων,
		\textit{ScyllaDB} για την αποθήκευση χρονοσειρών και
		\textit{Arroyo} για την επεξεργασία ροών, απέδειξε την
		ικανότητα της πλατφόρμας να διαχειρίζεται υψηλούς ρυθμούς
		εισαγωγής δεδομένων. Η αρχιτεκτονική είναι ανθεκτική σε
		διακοπές, θέτοντας τα θεμέλια για \textit{self-healing}
		συμπεριφορές.
	\item \textbf{Αποδοτικότητα και Αξιοπιστία στην Επεξεργασία Ροών:} Η
		αλυσίδα επεξεργασίας ροών (\textit{streaming pipeline})
		\textbf{EMQX $\rightarrow$ Kafka $\rightarrow$ Arroyo}
		λειτούργησε ως η κύρια ραχοκοκαλιά της πλατφόρμας. Το
		\textit{Kafka} διαχειρίστηκε με επιτυχία την \textbf{ανεξάρτητη
		επεκτασιμότητα} (\textit{horizontal scalability}) των
		\textit{producers} (EMQX) και των \textit{consumers}
		(Arroyo/ScyllaDB), διασφαλίζοντας \textit{back-pressure} και
		\textbf{μη-απώλεια} δεδομένων. Η χρήση του \textit{Arroyo} για
		την επεξεργασία των ροών επιβεβαίωσε τη δυνατότητα για
		\textbf{stateful} (διατήρώντας έτσι την κατάσταση) και
		\textbf{\textit{exactly-once} semantics} (ακριβώς μια φορά
		επεξεργασίας), γεγονός κρίσιμο για την αξιοπιστία των
		εξαγόμενων χρονοσειρών.
	\item \textbf{Βέλτιστη Απόδοση Υποσυστημάτων:} Τα βασικά υποσυστήματα
		του Nostradamus επέδειξαν εξαιρετικές επιδόσεις. Ειδικότερα, η
		βάση δεδομένων \textit{ScyllaDB} διατήρησε το $\text{P95
		latency}$ σε \textbf{εξαιρετικά χαμηλά επίπεδα} (τιμές
		$<20\,\text{ms}$) ακόμη και κάτω από τον προσομοιωμένο φόρτο
		$\mathbf{350\,\text{RPS}}$, επιβεβαιώνοντας την καταλληλότητά
		της για εφαρμογές χρονοσειρών με απαιτήσεις \textit{low
		latency}.
	\item \textbf{Κρισιμότητα του TTL και των Invalidations:} Η μεθοδολογία
		της ρύθμισης του \textit{TTL} και η παρακολούθηση του ρυθμού
		\textit{invalidations} απέδειξαν ότι η ενεργός διαχείριση της
		επικαιρότητας των δεδομένων μπορεί να γίνει χωρίς σημαντική
		επιβάρυνση. Το κόστος της συχνής εκκαθάρισης απορροφάται
		αποτελεσματικά, γεγονός που καθιστά το \textit{TTL} ένα
		στρατηγικό εργαλείο για τον έλεγχο της κατανάλωσης πόρων μνήμης
		και την εξασφάλιση του \textit{data freshness}.
	\item \textbf{Ανάλυση Επιδόσεων μέσω Metrics (Prometheus/Grafana):} Η
		υλοποίηση του πλαισίου \textbf{\textit{monitoring}} με
		\textit{Prometheus} και \textit{Grafana} επέτρεψε την
		\textbf{ποσοτική ανάλυση} και την ανάδειξη λεπτομερειών της
		απόδοσης του συστήματος. Ειδικότερα: 
		\begin{itemize}
			\item Η παρακολούθηση μηχανισμών όπως ο ρυθμός
				\textit{invalidations} (ως δείκτης κόστους
				εκκαθάρισης) επιβεβαίωσε την αποτελεσματική
				διαχείριση της επικαιρότητας των δεδομένων.
			\item Ο εντοπισμός του \textit{jitter} στο
				\textit{write latency} (που πιθανότατα
				οφείλεται σε \textit{GC} ή διεργασίες
				συντήρησης) παρείχε κρίσιμα \textit{insights}
				για τα υποκείμενα σημεία συμφόρησης που πρέπει
				να αντιμετωπιστούν για την περαιτέρω
				βελτιστοποίηση της πλατφόρμας.
		\end{itemize}
		Αυτή η μεθοδολογία τεκμηρίωσε την τρέχουσα απόδοση και έθεσε τη
		βάση για τη διαγνωστική ικανότητα.
	\item \textbf{Ολοκληρωμένο Stack Παρατηρησιμότητας:} Η στρατηγική
		επέκταση του πλαισίου με την ενσωμάτωση των υποσυστημάτων
		\textbf{Loki} για τη συλλογή \textbf{Logs} και \textbf{Tempo}
		με \textit{OpenTelemetry} για \textbf{Distributed Tracing}
		ολοκληρώνει το \textit{Observability Stack} (Metrics, Logs,
		Traces). Αυτή η τριπλή προσέγγιση είναι θεμελιώδης για την
		υποστήριξη \textbf{self-healing} συμπεριφορών και επιτυγχάνει:
		\begin{itemize}
			\item \textbf{Tracing:} Τον εντοπισμό και τη διάγνωση
				καθυστερήσεων σε επίπεδο υπηρεσίας,
				επιτρέποντας την παρακολούθηση ενός μεμονωμένου
				αιτήματος καθ' όλη τη διαδρομή της μέσα από τις
				κατανεμημένες υπηρεσίες, καθώς και τα επιμέρους
				στάδια των υπηρεσιών.
			\item \textbf{Root Cause Analysis (RCA):} Την άμεση
				συσχέτιση των καταγεγραμμένων συμβάντων
				(\textit{logs}) με τις ανώμαλες τιμές των
				\textit{metrics} και των \textit{traces},
				επιταχύνοντας δραστικά τον χρόνο εντοπισμού της
				ρίζας του προβλήματος.
		\end{itemize}
		Με αυτόν τον τρόπο, η πλατφόρμα Nostradamus αποκτά την πλήρη
		διαγνωστική ικανότητα που απαιτείται για τη λήψη
		αυτοματοποιημένων αποφάσεων.
	\item \textbf{Επικύρωση Λύσης Caching:} Οι δοκιμές στο
		\textit{/aggregate} endpoint έδειξαν ότι το ενσωματωμένο
		\textit{caching} λειτουργεί όπως αναμενόταν, επιτρέποντας την
		γρήγορη ανάκτηση των ίδιων μετρήσεων (\textit{high hit
		probability}) και μειώνοντας την επιβάρυνση στην υποκείμενη
		βάση δεδομένων.
\end{enumerate}
